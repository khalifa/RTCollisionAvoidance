\chapter{CONCLUSÕES E TRABALHOS FUTUROS}

Como proposto, foi desenvolvido um simulador em tempo real com o objetivo principal de minimizar as colisões entre carros e ambulâncias em um cruzamento. Através das características propostas para o simulador e as tarefas definidas para o sistema, foi possível acompanhar em tempo real o comportamento do fluxo de veículos pelas vias de um mapa e o funcionamento dos radares considerando alguns cenários de testes, tais como, tráfico leve (poucos carros), tráfico pesado (muitos carros). Cada situação foi oriunda de demanda sem horários específicos, onde a lógica de comando dos radares no modelo computacional foi acionada pelo controlador lógico programável. 

Foi possível observar e coletar em tempo real os dados do modelo de simulação e do sistema de controle. Assim, através de tabelas e gráficos, pôde-se avaliar o sistema e constatar que o modelo de simulação atendeu aos diferentes estímulos das variáveis relativas à dinâmica do sistema. Da mesma forma, o modelo permitiu avaliar o comportamento de cada uma das vias com radares e dos comandos de ativação/desativação oriundos do controlador. 
 
Portanto, espera-se que ............................................................................................................. 
também possam ser analisadas e testadas com a abordagem aqui 
apresentada.