\chapter{INTRODUÇÃO}

Esse relatório apresenta o trabalho desenvolvido na disciplina de Sistemas De Tempo Real ofertada no curso de Ciência da Computação da Universidade Federal de Minas Gerais (\sigla{UFMG}{Universidade Federal de Minas Gerais}).

\begin{figure}[H]
    \centering
    \includegraphics[scale=0.5,angle=0]{img/LUAR_HC_ELSA.png}
    \caption{Entidades envolvidas na pesquisa: LUAR, HC de Belo Horizonte e ELSA}
    \label{figLUAR_HC_ELSA}
\end{figure}

\section{Motivação}

O cenário que engloba o trânsito em ruas de uma cidade proporciona observações interessantes sobre os acontecimentos envolvidos nesse ambiente. Acidentes entre veículos não são raros e na maioria das vezes envolve erro humano que poderia ser evitado com implementações de comportamentos automáticos nas entidades mecânicas.
 
Uma situação ideal pode ser ilustrada com carros que agem automaticamente de acordo com as características que o trânsito proporciona, principalmente em cruzamentos ou situações de risco iminente. E, seguindo essa modelagem, um sistema de tempo real deve existir para que os componentes sejam monitorados e comandados de maneira eficaz e correta. 

A proposta desse trabalho é desenvolver um simulador de trânsito com características de tempo real. Tal simulador possui um sistema centralizado com a função de orientar os motoristas sobre qual a melhor decisão a ser tomado ao aproximar-se de um cruzamento. O objetivo nesse contexto é evitar colisões com outros veículos e, na medida do possível, minimizar o tempo de espera para atravessar um cruzamento com segurança.

\section{Objetivo}

\section{Organização do Trabalho}

No capítulo 2 são apresentadas as características do simulador, os conceitos do sistema em tempo real, suas restrições, as tarefas do sistema, o escalonamento e suas abordagens, ainda nesse capítulo também são apresentadas a modelagem e as informações técnicas sobre a implementação do sistema. O capítulo 3 apresenta os cenários de testes e os testes realizados para validar o escalonamento das tarefas e o funcionamento do simulador. Concluímos com o capítulo 5 que apresenta uma análise final sobre esse trabalho e também, apresenta propostas de trabalhos futuros.

Para melhor entendimento, o texto desse relatório está organizado da seguinte forma: A Seção 2 contextualiza o assunto e apresenta os conceitos envolvidos. A Seção 3 apresenta a metodologia aplicada à pesquisa. A Seção 4 mostra o desenvolvimento do trabalho. A Seção 5 apresenta os resultados obtidos e as eventuais discussões que surgiram ao longo do desenvolvimento. Por fim, na Seção 6 são expostas as conclusões obtidas e alguns temas para trabalhos futuros.